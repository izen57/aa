\section*{Введение}
\addcontentsline{toc}{section}{Введение}
Термин «матрица» применяется во множестве разных областей: от программирования до кинематографии.

Матрица - это математический объект, представляющий из себя набор упорядоченных чисел (целых, дробных или даже комплексных). Эти числа записываются, как правило в виде квадратной или прямоугольной таблицы, над которой можно совершать различные операции. Матрицы — очень важный математический инструмент, позволяющий решать множество задач от систем уравнений до оптимизации поставок.

Мы встречаемся с матрицами каждый день, так как любая числовая информация, занесённая в таблицу, уже в какой-то степени считается матрицей.

Целью работы работы является изучение и реализация алгоритмов умножения матриц, вычисление трудоёмкости этих алгоритмов. В данной лабораторной работе рассматривается стандартный алгоритм умножения матриц, алгоритм Винограда и модифицированный алгоритм Винограда.

Для достижения цели ставятся следующие задачи:
\begin{itemize}
    \item изучить классический алгоритм умножения матриц, алгоритм Винограда и модифицированный алгоритм Винограда;
    \item сравнить классический алгоритм умножения матриц, алгоритм Винограда и модифицированный алгоритм Винограда;
    \item выявить достоинства и недостатки рассмотренных алгоритмов;
    \item дать оценку трудоёмкости алгоритмов;
    \item реализовать рассмотренные алгоритмы;
    \item замерить время работы алгоритмов;
    \item описать и обосновать полученные результаты в отчёте о выполненной лабораторной
работе. 
\end{itemize}