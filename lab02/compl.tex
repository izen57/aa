\subsection{Вычисление трудоёмкости алгоритма}
Введём модель трудоёмкости для оценки алгоритмов.
\begin{enumerate}
  	\item Базовые операции стоимостью 1: +, -, *, /, =, ==, <=, >=, !=, +=, [], получение полей класса.
	\item Оценка трудоёмкости цикла: 
	
	f\_цикла = f\_инициализации + f\_сравнения + N * (f\_инкремента + f\_сравнения + f\_тела).
	
	\item Стоимость условного перехода возьмём за 0, стоимость вычисления условия остаётся. В условном операторе может возникнуть лучший и худший случаи по трудоёмкости в зависимости от выполнения условия и в зависимости от входных данных алгоритма.
\end{enumerate}

\subsection{Оценка трудоёмкости алгоритмов умножения матриц}
%\hfill
Оценка трудоёмкости дана согласно введённой выше модели вычислений.

\begin{enumerate}
	\item Стандартный алгоритм
	
	$$f=2+M(2+2+Q(2+2+N(2+8+1+1+1)))=13 \cdot$$
	$$\cdot MNQ+4MQ+4M+2 \approx 13 \cdot MNQ$$ 
		
    \item Алгоритм Винограда
        \begin{enumerate}
            \item Трудоёмкость алгоритма Винограда:\\
            \item Первый цикл: $\frac{15}{2} \cdot N  Q + 5 \cdot M + 2$ \\
            Второй цикл: $\frac{15}{2} \cdot M  N + 5 \cdot M + 2$\\
            Третий цикл: $13 \cdot M  N Q + 12 \cdot M Q + 4 \cdot M + 2$\\
            \item Условный переход:
            
            $\begin{bmatrix}
            2    &&, \text{лучший случай при чётном N}\\
            15 \cdot QM + 4 \cdot M + 4 &&, \text{худший случай}\\
            \end{bmatrix} $ \\
            \item Итого:
            
            $f = \frac{15}{2} \cdot M  N + \frac{15}{2} \cdot Q  N + 9 \cdot M + 8 +  5 \cdot Q + 13 \cdot M  N Q + 12 \cdot M Q +
            \begin{bmatrix}
            2    &&, \text{в лучшем случае}\\
            15 \cdot QM + 4 \cdot M + 4 &&, \text{в худшем случае}\\
            \end{bmatrix} $ \\
            $$f \approx 13 \cdot MNQ $$
        \end{enumerate}

    \item Оптимизированный алгоритм Винограда

        Введём оптимизации:
		\begin{enumerate}
		    \item замена оперции = на += или -=;
		    \item избавление от деления в условиях цикла (j < N, j += 2);
			\item заносение проверки на нечётность количества строк внутрь основных циклов;
			\item расчёт условия для последнего цикла один раз, а далее использование флага;
		\end{enumerate}
			
        Первый цикл: $4 \cdot N  Q + 4 \cdot M + 2$ \\
        Второй цикл: $4 \cdot M  N + 4 \cdot M + 2$\\
        Третий цикл: $9 \cdot M  N Q + 10 \cdot M Q + 4 \cdot M + 2$\\
        Условный переход:

        $\begin{bmatrix}
            2   &&, \text{лучший случай (при четном N)}\\
            10 \cdot QM &&, \text{худший случай}\\
        \end{bmatrix} $ \\

        Трудоёмкость оптимизированного алгоритма Винограда:
        \\$$f = 4 \cdot N  Q + 4 \cdot M + 2 + 4 \cdot M  N + 4 \cdot M + 2 + 9 \cdot M  N Q + 10 \cdot M Q + 4 \cdot M + 2 + $$
        $$ + \begin{bmatrix}
                    2   &&, \text{л.c}\\
                    10 \cdot QM &&, \text{х.c}\\
            \end{bmatrix} \approx 9 \cdot MNQ$$
	\end{enumerate}

\subsection{Вывод}
На основе теоретических данных, полученных из аналитического раздела были построены схемы требуемых алгоритмов и вычислены их трудоёмкости.