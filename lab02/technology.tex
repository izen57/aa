\section{Технологическая часть}

\subsection{Требования к вводу}
На вход подаются две матрицы. Операция умножения двух матриц выполнима только в том случае, если число столбцов в первом сомножителе равно числу строк во втором; в этом случае говорят, что матрицы согласованы.

\subsection{Требования к выводу}
Матрица (результат умножения матриц, поданых на вход).

\subsection{Требования к программе}
Две пустые матрицы - корректный ввод. Программа не должна аварийно завершаться.

\subsection{Выбор языка программирования}
Языком программирования был выбран C++.

Время работы алгоритмов было замерено с помощью функции GetCPUTime из <Windows.h>.

\subsection{Сведения о модулях программы}
Программа состоит из модуля main.cpp. В модуле main одна за другой вызываются подпрограммы каждого рассматриваемого алгоритма, а также подпрограммы подготови к использыванию этих алгоритмов, например создание и инициализация матриц.

На листингах \ref{lst:base}, \ref{lst:vin}, \ref{lst:vin_opt} представлен код алгоритмов.
\begin{lstlisting}[label=lst:base,caption=Классический алгоритм умножения матриц]
my_matrix classical_multiplication(my_matrix matrix1, my_matrix matrix2)
{
	my_matrix result
	{
		std::vector<std::vector<float>>(matrix1.n, std::vector<float>(matrix2.m, 0)),
		matrix1.n,
		matrix2.m
	};

	for (int i = 0; i < result.n; i++)
		for (int j = 0; j < result.m; j++)
			for (int k = 0; k < matrix1.m; k++)
				result.values[i][j] += matrix1.values[i][k] * matrix2.values[k][j];

	return result;
}
 \end{lstlisting}
 
\newpage
\begin{lstlisting}[label=lst:vin,caption=Алгоритм Винограда]
	my_matrix result
	{
		std::vector<std::vector<float>>(matrix1.n, std::vector<float>(matrix2.m, 0)),
		matrix1.n,
		matrix2.m
	};

	std::vector<float> row(matrix1.n, 0);
	for (int i = 0; i < matrix1.n; i++)
		for (int j = 0; j < matrix1.m / 2; j++)
			row[i] += matrix1.values[i][2 * j] * matrix1.values[i][2 * j + 1];

	std::vector<float> column(matrix2.m, 0);
	for (int i = 0; i < matrix2.m; i++)
		for (int j = 0; j < matrix1.m / 2; j++)
			column[i] += matrix2.values[2 * j][i] * matrix2.values[2 * j + 1][i];

	for (int i = 0; i < matrix1.n; i++)
		for (int j = 0; j < matrix2.m; j++)
		{
			result.values[i][j] = -row[i] - column[j];
			for (int k = 0; k < matrix1.m / 2; k++)
				result.values[i][j] += (matrix1.values[i][2 * k + 1] + matrix2.values[2 * k][j]) * (matrix1.values[i][2 * k] + matrix2.values[2 * k + 1][j]);
		}

	if (matrix1.m % 2)
		for (int i = 0; i < matrix1.n; i++)
			for (int j = 0; j < matrix2.m; j++)
				result.values[i][j] += matrix1.values[i][matrix1.m - 1] * matrix2.values[matrix1.m - 1][j];

	return result;
}
\end{lstlisting}
 
\newpage
  \begin{lstlisting}[label=lst:vin_opt,caption=Оптимизированный алгоритм Винограда]
my_matrix optimized_winograd_algorithm(my_matrix matrix1, my_matrix matrix2)
{
	my_matrix result
	{
		std::vector<std::vector<float>>(matrix1.n, std::vector<float>(matrix2.m, 0)),
		matrix1.n,
		matrix2.m
	};

	std::vector<float> row(matrix1.n, 0);
	for (int i = 0; i < matrix1.n; i++)
		for (int j = 1; j < matrix1.m; j += 2)
			row[i] -= matrix1.values[i][j] * matrix1.values[i][j - 1];

	std::vector<float> column(matrix2.m, 0);
	for (int i = 0; i < matrix2.m; i++)
		for (int j = 1; j < matrix1.m; j += 2)
			column[i] -= matrix2.values[j][i] * matrix2.values[j - 1][i];

	for (int i = 0; i < matrix1.n; i++)
		for (int j = 0; j < matrix2.m; j++)
		{
			result.values[i][j] += row[i] + column[j];
			for (int k = 1; k < matrix1.m; k += 2)
			{
				result.values[i][j] += (matrix1.values[i][k - 1] + matrix2.values[k][j]) * (matrix1.values[i][k] + matrix2.values[k - 1][j]);
				if (matrix1.m % 2)
					result.values[i][j] += matrix1.values[i][matrix1.m - 1] * matrix2.values[matrix1.m - 1][j];
			}
		}

	return result;
}
 \end{lstlisting}

\newpage
\subsection{Функциональные тесты}
	\begin{center}
	      Таблица 3.1 -- Функциональные тесты
	\end{center}
\begin{table}[H]
\label{tabular:functional_test}
 \begin{center}
\begin{tabular}{ | c | c | c | c |}
			\hline
			% {\centering GallusGallus \\ CC \\ t = 104}
			\textbf{Матрица1} & \textbf{Матрица2} & \multicolumn{1}{|p{3cm}|}{\textbf{Ожидаемый результат}} & \multicolumn{1}{|p{3cm}|}{\textbf{Фактический результат}}\\ \hline
			$\begin{bmatrix} 
   			1&2&3 \\
    			4&5&6 \\ 
   			7&8&9 \\ 
			\end{bmatrix}$ & 
			$\begin{bmatrix} 
   			1&2&3 \\
    			4&5&6 \\ 
   			7&8&9 \\ 
			\end{bmatrix}$ &
			$\begin{bmatrix} 
   			30&36&42 \\
    			66&81&96 \\ 
   			102&126&150 \\ 
			\end{bmatrix} $ &
			$\begin{bmatrix} 
   			30&36&42 \\
    			66&81&96 \\ 
   			102&126&150 \\ 
			\end{bmatrix} $\\
			\hline

			
			$\begin{bmatrix} 
   			1&1&1 \\
    			1&1&1 \\ 
   			1&1&1 \\ 
			\end{bmatrix}$ & 
			$\begin{bmatrix} 
   			1&1&1 \\
    			1&1&1 \\ 
			\end{bmatrix}$ &
			$\begin{matrix} 
   			\text{Матрицы не могут} \\
    			\text{быть перемножены} \\ 
			\end{matrix} $ &
			$\begin{matrix} 
   			\text{Матрицы не могут} \\
    			\text{быть перемножены} \\ 
			\end{matrix} $ \\
			\hline
			
			$\begin{bmatrix} 
   			1&2 \\
    			3&4 \\ 
   			5&6 \\ 
			\end{bmatrix}$ & 
			$\begin{bmatrix} 
   			1&2&3 \\
    			4&5&6 \\ 
			\end{bmatrix}$ &
			$\begin{bmatrix} 
   			9&12&15 \\
    			19&26&33 \\ 
   			29&40&51 \\ 
			\end{bmatrix} $ &
			$\begin{bmatrix} 
   			9&12&15 \\
    			19&26&33 \\ 
   			29&40&51 \\ 
			\end{bmatrix} $ \\
			\hline
		\end{tabular}
		
 \end{center}
\end{table} 

Все реализации алгоритмов на практике показали ожидаемый результат.

\subsection{Вывод}

В этом разделе была представлена реализация алгоритмов классического умножения матриц, алгоритма Винограда, оптимизирвоанного алгоритма Винограда. Тестирование показало, что алгоритмы реализованы правильно и работают корректно.